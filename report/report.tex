%% 
%% Copyright 2007, 2008, 2009 Elsevier Ltd
%% 
%% This file is part of the 'Elsarticle Bundle'.
%% ---------------------------------------------
%% 
%% It may be distributed under the conditions of the LaTeX Project Public
%% License, either version 1.2 of this license or (at your option) any
%% later version.  The latest version of this license is in
%%    http://www.latex-project.org/lppl.txt
%% and version 1.2 or later is part of all distributions of LaTeX
%% version 1999/12/01 or later.
%% 
%% The list of all files belonging to the 'Elsarticle Bundle' is
%% given in the file `manifest.txt'.
%% 

%% Template article for Elsevier's document class `elsarticle'
%% with numbered style bibliographic references
%% SP 2008/03/01

%% Use the option review to obtain double line spacing
%% \documentclass[authoryear,preprint,review,12pt]{elsarticle}

%% Use the options 1p,twocolumn; 3p; 3p,twocolumn; 5p; or 5p,twocolumn
%% for a journal layout:
%% \documentclass[final,1p,times]{elsarticle}
%% \documentclass[final,1p,times,twocolumn]{elsarticle}
%% \documentclass[final,3p,times]{elsarticle}
%% \documentclass[final,3p,times,twocolumn]{elsarticle}
%% \documentclass[final,5p,times]{elsarticle}
%% \documentclass[final,5p,times,twocolumn]{elsarticle}

%% The amsthm package provides extended theorem environments
%% \usepackage{amsthm}

\newif\ifDRAFT
\DRAFTfalse
%\DRAFTtrue

\ifDRAFT
\documentclass[review,number,sort&compress,12pt]{elsarticle}
    %% The lineno packages adds line numbers. Start line numbering with
%% \begin{linenumbers}, end it with \end{linenumbers}. Or switch it on
%% for the whole article with \linenumbers.
%% \usepackage{lineno}
    \usepackage{lineno}
    \newcommand{\picwidth}{0.8\textwidth}
\else 
  \documentclass[final,5p,times]{elsarticle}
    \newcommand{\picwidth}{0.48\textwidth}
\fi

% packages
%\usepackage{fontspec}
%\setmainfont{Gulliver}

\usepackage{amsmath}
\usepackage{amssymb}
\usepackage{bm}
\usepackage{braket}
\usepackage{booktabs}
\usepackage{graphicx}
\usepackage{xcolor}
\usepackage{tikz}
\usetikzlibrary{patterns}
\usepackage{subcaption}
\usepackage{url}
\usepackage{setspace}
\usepackage{diagbox} % generate diagonal divided cell in table
\usepackage{float}
\usepackage{graphicx}
\usepackage{subcaption}
\floatplacement{figure}{H}

\usepackage[linesnumbered,ruled]{algorithm2e}
\usepackage{algpseudocode}
%\usepackage{mathalfa}
%\usepackage{mathrsfs}
\DeclareMathOperator*{\argmin}{argmin}
%\DeclareMathAlphabet{\mathcal}{OT1}{pzc}{m}{bf}
\usepackage[imagesright]{rotating}
%\usepackage{subfigure}
\captionsetup[subfigure]{labelformat=simple,labelsep=colon}
\renewcommand{\thesubfigure}{fig\arabic{subfigure}}



% new commands
\newcommand{\EQ}[1]{Eq.~(\ref{#1})}                
\newcommand{\EQUATION}[1]{Equation~(\ref{eq:#1})} 
\newcommand{\TWOEQS}[2]{Eqs.~(\ref{eq:#1})~and~(\ref{eq:#2})}  
\newcommand{\TWOEQUATIONS}[2]{Equations~(\ref{eq:#1})~and~(\ref{eq:#2})}  
\newcommand{\EQS}[1]{Eqs.~(\ref{#1})}             %-- Eqs. (refeqs)
\newcommand{\EQUATIONS}[1]{Equations~(\ref{#1})}  %-- Eqs. (refeqs)
\newcommand{\FIG}[1]{Fig.~\ref{#1}}               %-- Fig. refig
\newcommand{\FIGURE}[1]{Figure~\ref{#1}}          %-- Figure refig
\newcommand{\TAB}[1]{Table~\ref{#1}}              %-- Table tablref
\newcommand{\SEC}[1]{Section~\ref{#1}}               %-- Eq. (refeq)
\newcommand{\REF}[1]{Ref.~\citen{#1}}               %-- Eq. (refeq)
\newcommand{\BLUE}[1]{\textcolor{blue}{#1}}
\renewcommand{\vec}[1]{\boldsymbol{#1}} %vector is bold italic
\newcommand{\vd}{\bm{\cdot}} % slightly bold vector dot
\newcommand{\grad}{\vec{\nabla}} % gradient
\newcommand{\ud}{\mathop{}\!\mathrm{d}} % upright derivative symbol
\graphicspath{{../data/rendered/}}


\ifDRAFT
\doublespacing
\fi

\journal{STAT 713}
\usepackage{filecontents,catchfile}

\begin{document}
% \sloppy % prevent words spill into the margin
\begin{frontmatter}

\title{Dynamic Mode Decomposition as a Linear Model for Spatiotemporal Systems}

\author{Jeremy A. Roberts\corref{cor}}
\ead{jaroberts@k-state.edu}
\address{Department of Statistics, Kansas State University, Manhattan, KS 66506, USA}
\cortext[cor]{Corresponding author}

% 100–300 word summary of your report
\begin{abstract}

\end{abstract}

\begin{keyword}
 spatio-temporal basis functions \sep rank reduction \sep dynamic mode decomposition.
\end{keyword}
\end{frontmatter}

\ifDRAFT
\linenumbers
\fi

%%%%%%%%%%%%%%%%%%%%%%%%%%%%%%%%%%%%%%%%%%%%%%%%%%%%%%%%%%%%%%%%%%%%%%%%%%%%%%%%
\section{Introduction}
\label{sec:introduction}
% 500--1000 words

Many of the largest data sets in this era of ``big data'' are generated from measurements of processes that evolve in space and time.
These measurements may be physical or simulated, but the often inseparable dependence of the latent process (i.e., the ``true'' but unknown process) on space and time makes the development of satisfactory statistical models for the data a challenging task.
A rich literature exists on spatio-temporal, statistical models (see, e.g., Refs. \cite{cressie2011sst} and \cite{wikle2019sts}).
Such models can be categorized as descriptive or dynamic.  
Descriptive models define a mean and covariance, often via kriging, and are consistent with the spirit of general linear models.  
Dynamic models assume that evolution in space and time follow some explicity defined, usually Markovian, process and, hence, appear to live somewhat outside the domain of general linear models (though the processes assumed can be and often are linear).

For both descriptive and dynamic models, the construction of suitable basis vectors in space, time, or both is often important for reducing the dimension of the problem.
With descriptive models, use of a low-rank basis, as in fixed-rank kriging, leads to smaller, more practical covariance matrices.
For dynamic models, the governing process can be projected onto a low-rank basis that captures the dominant characteristics of the system.
Common basis functions including linear, wavelet, Gaussian, and sinusoidal shapes in both space and time, and these can be combined via Kronecker products to produce a spatio-temporal basis.
Alternative basis vectors known as empirical orthogonal functions (EOFs) can be derived directly from the spatio-temporal observations through use of the singular-value decomposition of the space-averaged or time-averaged observations.
Such inherently discrete vectors can be used as continuous functions with appropriate interpolation.

The effort described here explored the use of dynamic-mode decomposition (DMD) \cite{schmid:hal-01053394} to produce spatially-discrete, temporally continuous basis functions for use in spatio-temporal, statistical models.
The DMD algorithm was first reported as a technique by which to analyzie spatio-temporal structures in complex fluid flows \cite{schmid:hal-01053394, schmid2010dynamic}.
Several extensions to DMD and its connection to the much older Koopman operator theory have been well described in a recent monograph \cite{kutzbook}.  
Basically, given a nonlinear, dynamical system
\begin{equation}
  \frac{d\mathbf{z}}{dt} = \mathbf{f}(\mathbf{z}) \, ,
  \label{eq:dynamicalsystem}
\end{equation}
the infinite dimensional but linear Koopman operator $\mathcal{K}$ satisfies the relationship $\mathcal{K}\mathbf{g}(\mathbf{z}) = \mathbf{g}(\mathbf{f}(\mathbf{z}))$ for some measurement function $\mathbf{g}$.
If the eigenfunctions of $\mathcal{K}$ are known, then measurements $\mathbf{g}(\mathbf{z})$ made from the original, nonlinear system can instead be defined directly using an eigenfunction expansion and known initial conditions.
In practice, neither $\mathcal{K}$ nor its eigenfunctions are generally accessible, and finite-dimensional approximations are sought; the DMD algorithm provides this approximation, and the resulting eigenmodes it produces form the basis vectors used in the present work.

To limit the scope of the study, the generated basis vectors were applied only within a standard, linear model (i.e., fixed effects with uncorrelated, normally-distributed errors). Moreover, the development is limited to those problems in which measurements $Z_{ij}$ are made on a fixed, regular, spatio-temporal grid, i.e., for spatial points $\mathbf{r}_i, i = 1, 2, \ldots m$ and times $t_j, j = 1, 2, \ldots n$. 
Here, each spatial coordinate $\mathbf{r}_i$ is treated as a vector to account for more than one dimension, e.g., $\mathbf{r}_i = [x_i, y_i, z_i]$ in three-dimensional, Cartesian coordinates.
In other words, the entire set of measurements can be viewed as a matrix $\mathbf{Z} \in \mathbb{R}^{m\times n}$.  
An alternative representation is the reshaped, $mn \times 1$ vector $\mathbf{z} = [Z_{11}, Z_{21}, \ldots Z_{m1}, Z_{12}, \ldots, Z_{mn}]$.
With this representation, let
\begin{equation}
 \mathbf{z} = \boldsymbol{\Psi}\boldsymbol{\beta} + \boldsymbol{\varepsilon} \, ,
\end{equation}
where $\boldsymbol{\varepsilon} \sim N(0, \mathbf{I}\sigma^2_{\varepsilon})$ and the columns $\boldsymbol{\psi}_k$ of the design matrix $\boldsymbol{\Psi}$ are basis vectors.

Although the DMD algorithm provides a deterministic way by which to define $\mathbf{\psi}$, there are many ways to choose how many modes to use, which modes to use, and how to combine the selected modes.
A key issue addressed in the study is to explore statistical metrics for choosing the number of basis functions to use and how to combine them in an optimal way.
To aid in this exploration, sea-surface temperature (SST) anomaly data as provided by Ref.~\cite{wikle2019sts} were adopted for all analyses.
Because only one set of data was explored, this study is naturally preliminary in nature, but the results suggest that DMD may be a very powerful way to incorporate the leading dynamics of a system into a basis used for fixed effects (as studied here) and, possibly, random effects in general linear models.


%%%%%%%%%%%%%%%%%%%%%%%%%%%%%%%%%%%%%%%%%%%%%%%%%%%%%%%%%%%%%%%%%%%%%%%%%%%%%%%%
\section{Methods}
\label{sec:methods}
% 500--1000 words

Central to the analysis presented here is the DMD algorithm.
Despite its connections to the more general Koopman theory, DMD can be motivated in somewhat simpler (albeit, heuristic) terms.
To motivate DMD, consider a dynamical system described by Eq.~(\ref{eq:dynamicalsystem}), where 
Let such a system be observed, leading to a sequence of observations of some discrete field in time, i.e., [$\mathbf{Z}_{:, 1}, \mathbf{Z}_{:. 2}, \ldots \mathbf{Z}_{:, n}$], where the $m$-vector $\mathbf{Z}_{:, j}$ is the $j$th column of matrix $\mathbf{Z}$. 
For simplicity, assume these observations are taken at equally-spaced times $t_0, t_1, \ldots t_n$ with $t_{i+1} = t_i + \Delta t$.  
Now, let there be some, possibly approximate, linear mapping $\mathbf{Z}_{:,j+1} \approx \mathbf{A}\mathbf{Z}_{:,j}$, or \begin{equation}
   \mathbf{Z}_+ \approx \mathbf{A}\mathbf{Z}_- \, ,
\end{equation}
where
\begin{equation}
  \mathbf{Z}_+ =  [\mathbf{Z}_{:,2},\mathbf{Z}_{:,3}, \ldots,\mathbf{Z}_{:,n}] \, ,
\end{equation}
and
\begin{equation}
 \mathbf{Z}_- =  [ \mathbf{Z}_{:,1},\mathbf{Z}_{:,2},\ldots, \mathbf{Z}_{:,n-1}] \,  .                                                                                                           
\end{equation}

The best-fit operator (in the Frobenius-norm sense) is $\mathbf{A}\approx \mathbf{Z}_+ \mathbf{Z}^{\dagger}_-$, where $\dagger$ indicates the pseudoinverse.  
In practice, the (thin) singular value decomposition
\begin{equation}
 \mathbf{Z}_-=\mathbf{U\Sigma V}^*  \, ,
\end{equation}
is used to define 
\begin{equation}
 \mathbf{Z}^{\dagger}_- = \mathbf{V}\boldsymbol{\Sigma}^{-1}\mathbf{U}^* \, ,
\end{equation}
where $\mathbf{U} \in \mathbb{C}^{m\times n}$, $\mathbf{V} \in \mathbb{C}^{n\times n}$, and $\boldsymbol{\Sigma} \in \mathbb{C}^{n\times n}$.  
Then the best-fit operator is
\begin{equation}
 \mathbf{A} = \mathbf{Z}_+ \mathbf{V} \boldsymbol{\Sigma}^{-1}\mathbf{U}^* \, ,
\end{equation}
but because this matrix can be intractably large, the low-rank approximation
\begin{equation}
 \tilde{\mathbf{A}} = \mathbf{U}^*_r \mathbf{A} \mathbf{U})_r 
\end{equation}
is used, where $\mathbf{U}_r$ contains the first $r < n$ columns of the left singular matrix $\mathbf{U}$, usually corresponding to the largest $r$ singular values.  


is used to define the reduced-rank 
\begin{equation}
\tilde{\mathbf{A}} = \mathbf{U}^* \mathbf{Z}_-\mathbf{V}\mathbf{\Sigma}^{-1} \, ,
\end{equation}
where $*$ indicates the complex transpose.  Finally, the eigendecomposition $\tilde{\mathbf{A}}\mathbf{W} = \mathbf{W}\boldsymbol{\Lambda}$ leads to $\boldsymbol{\phi} = \mathbf{X}_+ \mathbf{V}\boldsymbol{\Sigma}^{-1}\mathbf{W}$, which are (approximate) eigenvectors of $\mathbf{A}$ and the explicit-in-time representation $\mathbf{x}(t) = \sum_{k=1} \phi_k e^{\omega_k t}  \beta_k$, where $\omega_k = \ln{\lambda_k}/\Delta t$ and $\lambda_k$ is the $k$th diagonal element of $\boldsymbol{\Lambda}$.


%%%%%%%%%%%%%%%%%%%%%%%%%%%%%%%%%%%%%%%%%%%%%%%%%%%%%%%%%%%%%%%%%%%%%%%%%%%%%%%%
\section{Results}
\label{sec:results}
% 500--1000 words


%%%%%%%%%%%%%%%%%%%%%%%%%%%%%%%%%%%%%%%%%%%%%%%%%%%%%%%%%%%%%%%%%%%%%%%%%%%%%%%%
\section{Discussion}
\label{sec:discusion}
% 500--1000 words
% The discussion should place your results into the context of existing
% work and explore what additional research still needs to be done.


\section*{References}
\bibliographystyle{elsarticle-num} 
\bibliography{references}

\end{document}
