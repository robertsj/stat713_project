%% 
%% Copyright 2007, 2008, 2009 Elsevier Ltd
%% 
%% This file is part of the 'Elsarticle Bundle'.
%% ---------------------------------------------
%% 
%% It may be distributed under the conditions of the LaTeX Project Public
%% License, either version 1.2 of this license or (at your option) any
%% later version.  The latest version of this license is in
%%    http://www.latex-project.org/lppl.txt
%% and version 1.2 or later is part of all distributions of LaTeX
%% version 1999/12/01 or later.
%% 
%% The list of all files belonging to the 'Elsarticle Bundle' is
%% given in the file `manifest.txt'.
%% 

%% Template article for Elsevier's document class `elsarticle'
%% with numbered style bibliographic references
%% SP 2008/03/01

%% Use the option review to obtain double line spacing
%% \documentclass[authoryear,preprint,review,12pt]{elsarticle}

%% Use the options 1p,twocolumn; 3p; 3p,twocolumn; 5p; or 5p,twocolumn
%% for a journal layout:
%% \documentclass[final,1p,times]{elsarticle}
%% \documentclass[final,1p,times,twocolumn]{elsarticle}
%% \documentclass[final,3p,times]{elsarticle}
%% \documentclass[final,3p,times,twocolumn]{elsarticle}
%% \documentclass[final,5p,times]{elsarticle}
%% \documentclass[final,5p,times,twocolumn]{elsarticle}

%% The amsthm package provides extended theorem environments
%% \usepackage{amsthm}

\newif\ifDRAFT
\DRAFTfalse
%\DRAFTtrue

\ifDRAFT
\documentclass[review,number,sort&compress,12pt]{elsarticle}
    %% The lineno packages adds line numbers. Start line numbering with
%% \begin{linenumbers}, end it with \end{linenumbers}. Or switch it on
%% for the whole article with \linenumbers.
%% \usepackage{lineno}
    \usepackage{lineno}
    \newcommand{\picwidth}{0.8\textwidth}
\else 
  \documentclass[final,5p,times]{elsarticle}
    \newcommand{\picwidth}{0.48\textwidth}
\fi

% packages
%\usepackage{fontspec}
%\setmainfont{Gulliver}

\usepackage{amsmath}
\usepackage{amssymb}
\usepackage{bm}
\usepackage{braket}
\usepackage{booktabs}
\usepackage{graphicx}
\usepackage{xcolor}
\usepackage{tikz}
\usetikzlibrary{patterns}
\usepackage{subcaption}
\usepackage{url}
\usepackage{setspace}
\usepackage{diagbox} % generate diagonal divided cell in table
\usepackage{float}
\usepackage{graphicx}
\usepackage{subcaption}
\floatplacement{figure}{H}

\usepackage[linesnumbered,ruled]{algorithm2e}
\usepackage{algpseudocode}
%\usepackage{mathalfa}
%\usepackage{mathrsfs}
\DeclareMathOperator*{\argmin}{argmin}
%\DeclareMathAlphabet{\mathcal}{OT1}{pzc}{m}{bf}
\usepackage[imagesright]{rotating}
%\usepackage{subfigure}
\captionsetup[subfigure]{labelformat=simple,labelsep=colon}
\renewcommand{\thesubfigure}{fig\arabic{subfigure}}



% new commands
\newcommand{\EQ}[1]{Eq.~(\ref{#1})}                
\newcommand{\EQUATION}[1]{Equation~(\ref{eq:#1})} 
\newcommand{\TWOEQS}[2]{Eqs.~(\ref{eq:#1})~and~(\ref{eq:#2})}  
\newcommand{\TWOEQUATIONS}[2]{Equations~(\ref{eq:#1})~and~(\ref{eq:#2})}  
\newcommand{\EQS}[1]{Eqs.~(\ref{#1})}             %-- Eqs. (refeqs)
\newcommand{\EQUATIONS}[1]{Equations~(\ref{#1})}  %-- Eqs. (refeqs)
\newcommand{\FIG}[1]{Fig.~\ref{#1}}               %-- Fig. refig
\newcommand{\FIGURE}[1]{Figure~\ref{#1}}          %-- Figure refig
\newcommand{\TAB}[1]{Table~\ref{#1}}              %-- Table tablref
\newcommand{\SEC}[1]{Section~\ref{#1}}               %-- Eq. (refeq)
\newcommand{\REF}[1]{Ref.~\citen{#1}}               %-- Eq. (refeq)
\newcommand{\BLUE}[1]{\textcolor{blue}{#1}}
\renewcommand{\vec}[1]{\boldsymbol{#1}} %vector is bold italic
\newcommand{\vd}{\bm{\cdot}} % slightly bold vector dot
\newcommand{\grad}{\vec{\nabla}} % gradient
\newcommand{\ud}{\mathop{}\!\mathrm{d}} % upright derivative symbol
\graphicspath{{../data/rendered/}}


\ifDRAFT
\doublespacing
\fi

\journal{STAT 713}
\usepackage{filecontents,catchfile}

\begin{document}
% \sloppy % prevent words spill into the margin
\begin{frontmatter}

\title{Dynamic Mode Decomposition as a Linear Model for Spatiotemporal Systems}

\author{Jeremy A. Roberts\corref{cor}}
\ead{jaroberts@k-state.edu}
\address{Department of Statistics, Kansas State University, Manhattan, KS 66506, USA}
\cortext[cor]{Corresponding author}

% 100–300 word summary of your report
\begin{abstract}

\end{abstract}

\begin{keyword}
 spatio-temporal basis functions \sep rank reduction \sep dynamic mode decomposition.
\end{keyword}
\end{frontmatter}

\ifDRAFT
\linenumbers
\fi

%%%%%%%%%%%%%%%%%%%%%%%%%%%%%%%%%%%%%%%%%%%%%%%%%%%%%%%%%%%%%%%%%%%%%%%%%%%%%%%%
\section{Introduction}
\label{sec:introduction}
% 500--1000 words

Many of the largest data sets in this era of ``big data'' are generated from measurements of processes that evolve in space and time.
These measurements may be physical or simulated, but the often inseparable dependence of the latent process (i.e., the ``true'' but unknown process) on space and time makes the development of satisfactory statistical models a challenging task.
Although a rich literature exists on spatio-temporal statistics (see, e.g., Refs. \cite{cressie2011sst} and \cite{wikle2019sts}), the present effort focused specifically on the generation of spatio-temporal basis functions for use in a standard, linear model.  
Consider a set of measurements $\mathbf{Z} = [\mathbf{z}_1, \mathbf{z}_2, \ldots, \mathbf{z}_n]$
\begin{equation}
 a
\end{equation}



%%%%%%%%%%%%%%%%%%%%%%%%%%%%%%%%%%%%%%%%%%%%%%%%%%%%%%%%%%%%%%%%%%%%%%%%%%%%%%%%
\section{Methods}
\label{sec:methods}
% 500--1000 words


%%%%%%%%%%%%%%%%%%%%%%%%%%%%%%%%%%%%%%%%%%%%%%%%%%%%%%%%%%%%%%%%%%%%%%%%%%%%%%%%
\section{Results}
\label{sec:results}
% 500--1000 words


%%%%%%%%%%%%%%%%%%%%%%%%%%%%%%%%%%%%%%%%%%%%%%%%%%%%%%%%%%%%%%%%%%%%%%%%%%%%%%%%
\section{Discussion}
\label{sec:discusion}
% 500--1000 words
% The discussion should place your results into the context of existing
% work and explore what additional research still needs to be done.


\section*{References}
\bibliographystyle{elsarticle-num} 
\bibliography{references}

\end{document}
